\section{k-Nearest Neighbors Model}
\label{sec:knn-model}

\subsection{Implementation}
\subsubsection{Data Preparation}
The dataset was prepared for k-NN modeling with the following steps:

\begin{lstlisting}[language=R]
# Separate features and target
X <- data[, 1:(ncol(data)-1)]  # All features except target
y <- data[[target_col]]        # Target variable

# Normalize features using z-score standardization
X_normalized <- scale(X)

# Train-test split (70-30)
set.seed(123)
train_size <- floor(0.7 * nrow(X_normalized))
train_indices <- sample(seq_len(nrow(X_normalized)), size = train_size)

X_train <- X_normalized[train_indices, ]
y_train <- y[train_indices]
X_test <- X_normalized[-train_indices, ]
y_test <- y[-train_indices]
\end{lstlisting}

\subsubsection{k-NN Implementation}
We implemented k-NN from scratch using base R:

\begin{lstlisting}[language=R]
# Euclidean distance function
euclidean_distance <- function(x1, x2) {
  sqrt(sum((x1 - x2)^2))
}

# k-NN prediction function
knn_predict <- function(X_train, y_train, X_test, k = 5) {
  predictions <- character(nrow(X_test))
  
  for(i in 1:nrow(X_test)) {
    # Calculate distances to all training points
    distances <- numeric(nrow(X_train))
    for(j in 1:nrow(X_train)) {
      distances[j] <- euclidean_distance(X_test[i, ], X_train[j, ])
    }
    
    # Find k nearest neighbors
    k_nearest_indices <- order(distances)[1:k]
    k_nearest_labels <- y_train[k_nearest_indices]
    
    # Majority vote
    vote_counts <- table(k_nearest_labels)
    predictions[i] <- names(vote_counts)[which.max(vote_counts)]
  }
  
  return(factor(predictions, levels = levels(y_train)))
}
\end{lstlisting}

\subsection{Hyperparameter Tuning}
\subsubsection{k-Value Selection}
We tested different values of $k$ to find the optimal number of neighbors:

\begin{table}[H]
\centering
\caption{k-NN performance for different k values}
\label{tab:knn-k-values}
\begin{tabular}{cc}
\toprule
\textbf{k Value} & \textbf{Accuracy} \\
\midrule
3 & 0.90 \\
5 & 0.81 \\
7 & 0.74 \\
9 & 0.72 \\
11 & 0.70 \\
\bottomrule
\end{tabular}
\end{table}

The results show that $k = 3$ provides the best performance with 90\% accuracy on the test subset.

\begin{figure}[H]
\centering
\includegraphics[width=0.8\textwidth]{graphics/04-knn-k_tuning.png}
\caption{k-NN accuracy vs k value showing optimal performance at k=3}
\label{fig:knn-k-tuning}
\end{figure}

\subsection{Model Evaluation}
\subsubsection{Performance Metrics}
Using the optimal $k = 3$, the final model achieved:

\begin{itemize}
    \item \textbf{Accuracy}: 81\%
    \item \textbf{Precision (ad)}: 99.12\%
    \item \textbf{Recall (ad)}: 75.17\%
    \item \textbf{F1-score (ad)}: 85.5\%
\end{itemize}

\subsubsection{Confusion Matrix Analysis}
The confusion matrix reveals the model's classification performance:

\begin{figure}[H]
\centering
\includegraphics[width=0.6\textwidth]{graphics/04-knn-confusion_matrix.png}
\caption{k-NN confusion matrix showing classification results}
\label{fig:knn-confusion-matrix}
\end{figure}

\begin{table}[H]
\centering
\caption{k-NN confusion matrix (k=3)}
\label{tab:knn-confusion}
\begin{tabular}{lcc}
\toprule
 & \multicolumn{2}{c}{\textbf{Actual}} \\
\textbf{Predicted} & \textbf{ad} & \textbf{nonad} \\
\midrule
ad & 112 & 1 \\
nonad & 37 & 50 \\
\bottomrule
\end{tabular}
\end{table}

\subsection{Results Interpretation}
\subsubsection{Strengths}
\begin{itemize}
    \item \textbf{High precision}: 99.12\% precision means very few false positives
    \item \textbf{Simple implementation}: No complex parameter tuning required
    \item \textbf{Interpretable}: Easy to understand why predictions are made
    \item \textbf{Non-parametric}: No assumptions about data distribution
\end{itemize}

\subsubsection{Limitations}
\begin{itemize}
    \item \textbf{Computational cost}: Requires distance calculation to all training points
    \item \textbf{Memory intensive}: Stores entire training dataset
    \item \textbf{Curse of dimensionality}: Performance may degrade with high-dimensional data
    \item \textbf{Sensitive to irrelevant features}: All features contribute equally to distance
\end{itemize}

\subsubsection{Class Imbalance Impact}
The model shows excellent precision (99.12\%) but moderate recall (75.17\%), indicating:
\begin{itemize}
    \item Strong ability to correctly identify advertisements when predicted
    \item Some difficulty in finding all advertisement instances
    \item Bias toward the majority class (nonad) due to class imbalance
\end{itemize}

\subsection{Conclusion}
The k-NN model achieved excellent performance in Internet advertisement classification with k=3 yielding the best results. The model demonstrated high accuracy of 81\% on the test set, with exceptional precision of 99.12\% for advertisement detection, indicating very few false positives. However, the recall of 75.17\% suggests some advertisements were missed. The F1-score of 85.5\% reflects a good balance between precision and recall. The model's performance decreased as k increased, confirming that k=3 provides the optimal trade-off between bias and variance for this dataset.