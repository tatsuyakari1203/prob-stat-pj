\section{Descriptive Statistics}
\label{sec:descriptive-statistics}

This section presents a comprehensive exploratory data analysis (EDA) of the Internet Advertisements dataset to understand the data distribution, relationships between variables, and key characteristics that will inform our modeling approach.

\subsection{Target Variable Analysis}
The target variable distribution reveals a significant class imbalance in the dataset:
\begin{itemize}
    \item Advertisement images (ad): 459 observations (14\%)
    \item Non-advertisement images (nonad): 2,820 observations (86\%)
\end{itemize}

This 1:6 ratio between advertisement and non-advertisement classes is typical in real-world advertisement detection scenarios and must be considered when evaluating model performance.

\begin{figure}[H]
\centering
\includegraphics[width=0.7\textwidth]{graphics/03-eda-target_distribution.png}
\caption{Distribution of target variable showing class imbalance}
\label{fig:target-distribution}
\end{figure}

The target distribution visualization was generated using:

\begin{lstlisting}[language=R]
# Bar plot for target variable distribution
target_table <- table(data[[target_col]])
barplot(target_table, 
        main = "Distribution of Target Variable",
        xlab = "Class", ylab = "Frequency",
        col = c("lightcoral", "lightblue"),
        border = "black")
\end{lstlisting}

\subsection{Feature Distribution Analysis}
To understand the characteristics of individual features, we analyzed the distribution of the first 10 features in the dataset. The summary statistics reveal varying scales and distributions across features:

\begin{table}[H]
\centering
\caption{Summary statistics for the first 10 features}
\label{tab:feature-stats}
\begin{tabular}{lrrrrrrr}
\toprule
\textbf{Feature} & \textbf{Min} & \textbf{Q1} & \textbf{Median} & \textbf{Mean} & \textbf{Q3} & \textbf{Max} & \textbf{SD} \\
\midrule
X0 & 1.00 & 32.50 & 51.00 & 60.44 & 61.00 & 640 & 47.06 \\
X1 & 1.00 & 90.00 & 110.00 & 142.89 & 144.00 & 640 & 112.56 \\
X2 & 0.00 & 1.28 & 2.10 & 3.41 & 3.90 & 60 & 5.20 \\
X3 & 0.00 & 1.00 & 1.00 & 0.77 & 1.00 & 1 & 0.42 \\
X4 & 0.00 & 0.00 & 0.00 & 0.004 & 0.00 & 1 & 0.065 \\
\bottomrule
\end{tabular}
\end{table}

Histograms for the first six features show diverse distribution patterns:

\begin{figure}[H]
\centering
\includegraphics[width=\textwidth]{graphics/03-eda-histograms.png}
\caption{Histograms of the first six features showing distribution patterns}
\label{fig:histograms}
\end{figure}

The histogram generation code:

\begin{lstlisting}[language=R]
# Histograms for the first 6 features
par(mfrow = c(2, 3))
for(i in 1:6) {
  hist(data[[i]], 
       main = paste("Histogram of Feature", i),
       xlab = paste("Feature", i),
       ylab = "Frequency",
       col = "lightblue",
       border = "black",
       breaks = 30)
}
\end{lstlisting}

\subsection{Class-wise Feature Analysis}
To understand how features differ between advertisement and non-advertisement classes, we created boxplots comparing the distribution of key features across classes:

\begin{figure}[H]
\centering
\includegraphics[width=\textwidth]{graphics/03-eda-boxplots.png}
\caption{Boxplots of features 1 and 2 grouped by target class}
\label{fig:boxplots}
\end{figure}

The boxplot analysis was implemented as:

\begin{lstlisting}[language=R]
# Boxplots of features 1 and 2, grouped by target class
par(mfrow = c(1, 2))
boxplot(data[[1]] ~ data[[target_col]], 
        main = "Feature 1 by Class",
        xlab = "Class", ylab = "Feature 1 Value",
        col = c("lightcoral", "lightblue"))
boxplot(data[[2]] ~ data[[target_col]], 
        main = "Feature 2 by Class",
        xlab = "Class", ylab = "Feature 2 Value",
        col = c("lightcoral", "lightblue"))
\end{lstlisting}

\subsection{Feature Relationships and Scatter Analysis}
Scatter plots and density plots provide insights into feature relationships and class separability:

\begin{figure}[H]
\centering
\includegraphics[width=\textwidth]{graphics/03-eda-scatter_density_plots.png}
\caption{Scatter plots and density plots showing feature relationships and class distributions}
\label{fig:scatter-density}
\end{figure}

The visualization combines scatter plots and density analysis:

\begin{lstlisting}[language=R]
# Scatter plots and density plots
par(mfrow = c(2, 2))
colors <- c("red", "blue")
class_colors <- colors[as.numeric(data[[target_col]])]
plot(data[[1]], data[[2]], 
     main = "Feature 1 vs Feature 2", 
     xlab = "Feature 1", ylab = "Feature 2", 
     col = class_colors, pch = 16)
legend("topright", legend = levels(data[[target_col]]), 
       col = colors, pch = 16)
\end{lstlisting}

\subsection{Correlation Analysis}
Correlation analysis reveals important relationships between features. The correlation heatmap for the first 10 features shows the strength of linear relationships:

\begin{figure}[H]
\centering
\includegraphics[width=0.8\textwidth]{graphics/03-eda-correlation_heatmap.png}
\caption{Correlation heatmap of the first 10 features}
\label{fig:correlation-heatmap}
\end{figure}

The correlation analysis identified several perfect correlations (correlation = 1.0) between different feature pairs, indicating potential redundancy in the dataset:
\begin{itemize}
    \item Features X11 and X14: Perfect positive correlation
    \item Features X8 and X15: Perfect positive correlation
    \item Features X13 and X38: Perfect positive correlation
    \item Features X44 and X46: Perfect positive correlation
\end{itemize}

The correlation matrix was computed using:

\begin{lstlisting}[language=R]
# Correlation heatmap for the first 10 numeric features
numeric_data <- data[, sapply(data, is.numeric)]
cor_matrix <- cor(numeric_data[, 1:10], use = "complete.obs")
heatmap(cor_matrix, 
        symm = TRUE, 
        main = "Correlation Heatmap of First 10 Features",
        col = colorRampPalette(c("blue", "white", "red"))(100))
\end{lstlisting}

\subsection{Key Findings from Descriptive Analysis}
The exploratory data analysis reveals several important characteristics:

\begin{enumerate}
    \item \textbf{Class Imbalance}: The dataset has a significant imbalance with 86\% non-advertisements
    \item \textbf{Feature Diversity}: Features show diverse scales and distributions, suggesting the need for normalization
    \item \textbf{Perfect Correlations}: Multiple feature pairs show perfect correlation, indicating potential redundancy
    \item \textbf{Class Separability}: Some features show different distributions between advertisement and non-advertisement classes
    \item \textbf{High Dimensionality}: With 1,559 features, dimensionality reduction techniques may be beneficial
\end{enumerate}

These findings will inform our modeling approach, particularly regarding feature selection, data preprocessing, and evaluation metrics that account for class imbalance.